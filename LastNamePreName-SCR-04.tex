\documentclass[journal]{IEEEtran}
\usepackage{blindtext}

\usepackage{fancyhdr}
\usepackage{extramarks}
\usepackage{amsmath}
\usepackage{amsthm}
\usepackage{amsfonts}
\usepackage{tikz}
\usepackage[plain]{algorithm}
\usepackage{algpseudocode}
\usepackage[ngerman]{babel}

\usepackage[utf8]{inputenc}
\usepackage{amsmath}
\usepackage{amsthm}
\usepackage{amsfonts}
\usepackage{amssymb}
\usepackage{graphicx}

\usepackage{verbatim}

\ifCLASSINFOpdf
\else
\fi

\hyphenation{op-tical net-works semi-conduc-tor}

\begin{document}
\title{"'Sollte man in unserem Gesundheitssystem verstärkt eHealth nutzen?"}
\author{Review von Vorname~Nachname, Technische Universität Graz}

% The paper headers
    \markboth{"'Sollte man in unserem Gesundheitssystem verstärkt eHealth nutzen?" von S. Bürscher, No.~1, June~2015}%
    {Shell \MakeLowercase{\textit{et al.}}: Bare Demo of IEEEtran.cls for Journals}

% make the title are
\maketitle

%\begin{abstract}
%    Kernthematik 
    
    %\blindtext[1]
%\end{abstract}

%\begin{IEEEkeywords}
 %   Review, Smart Machines, D. Stoeckl.
%\end{IEEEkeywords}

\IEEEpeerreviewmaketitle
%\section{Einführung}
S. Bürscher stellt in seinem Beitrag zum Thema "'Sollte man in unserem Gesundheitssystem verstärkt 
eHealth nutzen?\grqq in Hinblick auf unsere Gesellschaft folgende Kernthesen auf: 
EHealth kann große positive Auswirkungen auf den Menschen haben, jedoch  auch Großkonzerne 
könnten Vorteile nutzen. Riesige Datenbanken könnten genutzt werden, damit die Wissenschaft 
bessere Schlüsse ziehen kann. Chronische Behandlungen könnten besser behandelt werden. 
Nichtsallzutrotz, gibt es viele Risikos, ob diese Daten nicht einfach weiterverkauft werden 
und nicht für den Menschen zum Vorteil genutzt werden. \\

Man verspricht sich eine Verbesserung der Gesundheit der Menschen. Es werden laufend Datenberge
auf einer Datenbank gespeichert, die der Arzt abrufen kann. So kann man bei kleinen Beschwerden bzw. 
Tendenzen schon Prognosen stellen. Bis heute war es immer so, dass erst bei Veränderungne des Gesundheitszustandes, wo der Patient beschwerden hat, etwas dagegen getan wird. Herz-Kreislauferkrankungen sind die häufigsten Erkrankungen, mit rund 45$\%$ \cite{bmv:herz}, die häufigste in Österreich.
Aus wirtschaftlicher Sicht unseres Sozialsystems, wäre das ein riesiger Vorteil, da Patienten 
mit dieser Predictive Healthcare frühzeitig versorgt werden könnten und nicht erst ins Spital 
landen, was ungeheure Umkosten verursachen. Ebenso die Menschen könnten ein gesünderes Leben führen, da die Daten mit ähnlichen Krankheitserscheinungen von anderen Patienten verglichen können werden. 
Laut Dr. Schreier \cite{schreier:ehealth} würden Neugeborene im Durchschnitt bis zu 105 Jahre alt werden.
Da fragt man sich, wie hoch das Pensionsalter steigen wird, damit man die Kosten von den Rentner decken kann. Mittlerweile gibt es mehr alte Leute als junge und die Frage, die ich mir Stelle ist, wie lange kann das noch so weitergehen. Man merkt, dass der Staat bereits viele Einsparungen vornimmt. Familienbeihilfe wird nur mehr bis zum 24. Lebensjahr \cite{help:gv} ausbezahlt. Offensichtlich werden 
die jungen Leute dazu gedrängt, sich für einen Beruf zu entscheiden, als studieren zu gehen. 
Eine weiterer Nebenaspekt, der sich eröffnen würde, wenn man die Predictive Healthcare einführen würde, ist dass man seine eigene Daten immer online abfragen kann und daher seinen eigenen Gesundheitszustand überwachen kann. Einige Leute hätten dann den Anreiz, mehr auf ihre Gesundheit zu achten und eventuell als Rentner ein sorgenfreieres Leben führen, ohne viele Krankenaufenthalte, wenn man
ständig seine Zukunft vor Augen hat. \\


Der Verfasser des Berichts liefert weitere gute Argumente, warum eHealth eingesetzt soll werden. 
Lange Wartezeiten bei den Hausärzten würden zur Vergangenheit angehören, wenn das ehealth System 
schon bereits genommene Medikamente wieder empfiehlt und damit sie gleich bei der nächsten Apotheke abgeholt werden können. Viele Leute hierzulande sind genervt von unserem langsamen Beamtensystem \cite{buerokratie:asterix}. 
Nicht allzuoft, muss man sich einen Tag frei nehmen damit man an den unpassenden Öffnungszeiten der Hausärzte gerecht wird. Sie sind die erste Anlaufstelle, wenn man ein Problem hat und dann wird man weiterüberwiesen an einem Facharzt, der dann helfen kann. Warum sollte man immer über eine Ecke laufen, 
damit man zum Arzt kommt? Dann muss man dort wieder einen Termin vereinbaren und eventuell Monate auf 
so manchen Fachärzten warten. \\

Die Akzeptanz von dieser neuen Reform von unserem Gesundheitssystem lässt sich noch streiten. Leute 
sind immer skeptisch gegenüber Veränderung, sowie das unsere Daten an Dritte weitergegeben werden. Wenn man das Bundesheer neu strukturieren \cite{bundesheer:reform} wollte, hat der ganze Medientummel dazu geführt, dass dies nicht geschehen ist. 
Wie bereits oben erwähnt, gehören ältere Menschen zur Mehrheit an und nach Volksabstimmungen wird Ihre Meinung überwiegen. Es ist generell bekannt, dass sie nicht mehr Lernen möchten, wie man einen Computer bedient, jedoch ist es schon in vielen Alltagssituation, wie beim Ticketdruck am Automat der ÖBB oder anderen Dienstleistungen \cite{welt:reform} sind kaum wegzudenken. 
Dennoch wehren sich Senioren es beizubringen, obwohl es allgegenwärtig ist. Man müsste warten, bis die Generationen im höhren Alter sind, die mit dem Computer aufgewachsen sind und keine Ablehnung gegen den Computer empfinden. \\



Abschließend muss man sagen, dass der Autor alle Facetten des Themas angeschnitten hat, jedoch ist es 
sehr komplex und man kann die Zukunft nicht vorhersagen. Deswegen ist es mir schwer gefallen etwaige Fehler zu finden. Wie sich die Zukunft verändern wird kann man nur abwarten und man kann sich sicher seien, dass es immer jemanden gibt der ein System ausnutzen möchte. Die Angst vor möglichen Datendiebstahl ist dennoch groß, aber warum sollte man seine Gesundheit auf das Spiel setzen. 




%\subsection{Hinterfragung der Andwendug}

\appendices

\ifCLASSOPTIONcaptionsoff
 \newpage
\fi


\begin{thebibliography}{1}
    \bibitem[GRIEBLER, ANZENBERGER, EISENMANN, 2015]{bmv:herz}
    GRIEBLER, Robert, ANZENBERGER, Judith, EISENMANN, Alexander.
    \emph{Herz-Kreislauf-Erkrankungen in Österreich}. Kopierstelle des BMG. 2015.

    \bibitem[SCHREIER, 2015]{schreier:ehealth}
     \begin{verbatim}https://www.youtube.com/watch?v=5LHm92mYHyQ \end{verbatim}
    
    \bibitem[ASTERIX, 2009]{buerokratie:asterix}
    \begin{verbatim} https://www.youtube.com/watch?v=lIiUR2gV0xk \end{verbatim}
    
    \bibitem[HELP.GV.AT, 2014]{help:gv}
    \begin{verbatim}https://www.help.gv.at/Portal.Node/hlpd/public/
    content/8/Seite.080714.html\end{verbatim}. 2014.
    
    \bibitem[SIMPER, 2013]{bundesheer:reform}
    SIMPER, Gerald. \emph{Bericht zur Reform des Wehrdienstes}. BMLVS/Heeresdruckzentrum 13-8314 |2702/13. 2013.

    \bibitem[Birger Nicolai, 2013]{welt:reform}
    BIRGER, Nicolai. \emph{Wie die moderne Welt alte Menschen diskriminiert}. Die Welt. 2013.
    
\end{thebibliography}
\end{document}


